\documentclass[10pt]{article}
\usepackage{NotesTeXV3}
\usepackage{float}
\usepackage{cleveref}

\title{Summary:\\
    Introduction into Deep Learning (IN2346)
    \\
    \large{\textit{The Art of Teaching Sands to Think\textsuperscript{TM}}}
    \\\vspace{0.5em}
    \small{Based on lecture of Prof. Niessner}
    }
\author{Yudhistira Arief Wibowo}

\begin{document}

\maketitle

\newpage 

\section*{Preface}

Hi, I made this note in hope to make me reminds myself in case I forget or have to re-learn I2DL. 
The content of this note is mostly taken from the lecture presentation from Prof. Niessner for 
the lecture IN2346 at TUM (as summer 2023). However, there might be slight mistakes in the content.
In case of that, please contact me. This note also shall be open-source available, thus you can 
create a merge request on the repository of this note.

Throughout the note, we will use several symbols. Similar to the lecture, we will use bold symbol 
to represent a matrix of its corresponding normal symbol. For example, $\pmb\theta$ is the matrix 
that contains bunch of values $\theta$.

\newpage

\section{Introduction}

Before beginning with the concept of deep learning, we want
to familiarize ourselves with the underlying concept of intelligence
itself. In the area of intelligence, there are 3 big terminologies
that categorize it. They are:


\begin{enumerate}
    \item \textbf{Artificial Intelligence}: Broad definitions, covers 
        any "smart" algorithm including \texttt{if} statements, 
        binary search, Dijkstra algorithm, etc. 
    \item \textbf{Machine Learning}: Performing learn on data, includes 
        linear regression, logistic regression, support vector 
        machines, etc.
    \item \textbf{Deep Learning}: Use neural networks, includes multi-layer
        perceptons, CNN, RNN, Transformer, etc. 
\end{enumerate}

\begin{marginfigure}
    \begin{figure}[H]
        \centering
        \includegraphics[width=\textwidth]{figures/learning_venn.pdf}
        \caption{The Venn diagram describing the coverage of each term}
    \end{figure}
\end{marginfigure}

\newpage

\section{Machine Learning}

Machine learning able to perform several types of task. These tasks 
usually can be divided into two types of learning methods:

\begin{enumerate}
    \item \textbf{Supervised} learning: The dataset contains the ground 
        truth label. Examples are:
        \begin{itemize}
            \item Regression 
            \item Classification
        \end{itemize}
    \item \textbf{Unsupervised} learning: The dataset does \textbf{not}
        contain ground truth label. Examples are:
        \begin{itemize}
            \item (k-means) Clustering
            \item Principal Component Analysis (PCA)
        \end{itemize}
\end{enumerate}

The one that will perform the task is called a \textbf{model} $M$. It 
will train itself using the \textbf{dataset}. This dataset 
contains \textbf{data points} $x$ and may have \textbf{ground truth label}
$y$. $M$ performs mathematical calculation on the dataset. $M$ 
has its own \textbf{parameters} $\theta$, which define the mathematical behavior of 
its component.

$M$ needs to learn based on the given input the most suitable output for given 
task. But how it will be able to learn at all? For simple task, such as linear 
regression, we can compute the solution using a certain formula. However,
more complex tasks requires several steps to be performed 
multiple times to approximate the result, which is called 
\textbf{iterative method}.

In iterative method, we need to update our model to 
improve its performance. This is called \textbf{optimization}. 
But what change shall be made to the model? To know this we need 
a metric called \textbf{loss function}. Loss function measures 
how good the estimation of the model and tells the optimization 
method how to make it better.

\subsection{Regressions}

A regression task is a task where the model shall \textbf{predict a 
continuous output} value (e.g., temperature of a room). In the lecture, 
we discussed a regression algorithm, it is \textbf{linear} regression.
In linear regression, we want to \textbf{find a linear model} that explains a target $y$
given inputs $x$. 

\begin{definition}
    For input $x_i$ with a dimension of $d$ and features $x_{ij}$, a linear model with 
    weights $\theta_j$ and bias $\theta_0$ for producing prediction $y_i$ is expressed in the form 
    \begin{equation*}
        \hat{y}_i = \sum_{j=1}^d x_{ij} \theta_j + \theta_0
    \end{equation*}
    or in matrix form 
    \marginnote{The additional 1 in each row in the matrix $\mathbf{X}$ exists to be multiplied with the bias $\theta_0$.}
    \begin{equation*}
        \mathbf{\hat{y}} = \mathbf{X} \pmb{\theta} \quad \text{or} \quad
        \left(\begin{matrix}
            \hat{y}_1 \\ \hat{y}_2 \\ \vdots \\ \hat{y}_n
        \end{matrix}\right) = 
        \left(\begin{matrix}
            1 & x_{11} & \dots & x_{1d} \\
            \vdots & \vdots & \ddots & \vdots \\
            1 & x_{n1} & \dots & x_{nd}
        \end{matrix}\right) 
        \left(\begin{matrix}
            \theta_0 \\ \theta_1 \\ \vdots \\ \theta_d
        \end{matrix}\right)
    \end{equation*}
\end{definition} 

To know how good or bad we fit our linear model to the dataset, 
we use loss function.

\begin{definition}
    For a linear regression with parameters $\pmb\theta$, the loss function $J$ 
    is defined as follows:
    \begin{equation*}
        J(\pmb\theta) = \frac{1}{n} \sum_{i=1}^n (\hat{y}_i - y_i)^2
    \end{equation*}
\end{definition}

In order to optimize the model, we need to find the correct parameter $\theta$ 
that minimize the value $J(\pmb\theta)$. This problem is called the \textbf{Linear 
Least Squares Estimates}. Fortunately, this is a \textbf{convex 
problem}, meaning that there exists a closed-form unique solution. Hence,
we don't need any iterative method.

\begin{definition}
    For a loss function $J(\pmb\theta)$, the value of $\theta$ that minimizes
    the loss is 
    \begin{equation*}
        \arg\min_\theta J(\pmb\theta) = (\mathbf{X}^T\mathbf{X})^{-1}\mathbf{X}^T\mathbf{y} 
    \end{equation*} 
\end{definition}

The lost function calculated before is also the same as finding the
\textbf{Maximal Likelihood Estimate} (given our training samples are independent
and equally distributed).

\subsection{Classification}

A classification algorithm, as the name describe, is an algorithm that 
tries to classify a certain data into two or more categories. In the lecture,
we discuss the \textbf{logistic regression}\mn{
    Despite its name, the logistic regression is a classification 
    algorithm (and not a regression algorithm in context of machine learning).
}
algorithm to help us
perform classification over 2 classes (e.g., a cat or not a cat). 
Since it can classify into exactly two classes, we call it as 
a \textbf{binary classification}.

A logistic regression algorithm uses the \textbf{sigmoid function} $\sigma$ 
to map values of the data into value between zero and one. 

\begin{definition}
    A sigmoid function $\sigma$ for a value $x$ is defined as follows:
    \begin{equation*}
        \sigma(x) = \frac{1}{1 + e^{-x}}
    \end{equation*}
\end{definition}

Since the value of the sigmoid function is between zero and one, we can 
interpret this value as a probability. To use this sigmoid function 
for our classification task, we perform weighted sum of the 
data $\mathbf{X}$ (with our weights $\pmb{\theta}$)
\begin{marginfigure}
    \begin{figure}[H]
        \includegraphics{figures/logistic_regression.pdf}
        \caption{An example of logistic regression}
    \end{figure}
\end{marginfigure}
and use this 
as the input of the sigmoid function.

As a performance measurement, we can use the \textbf{Binary 
Cross-Entropy (BCE) loss} function.

\begin{definition}
    The BCE loss of a prediction $\hat{y}_i$ for ground truth label $y_i$ 
    is defined as follows:
    \begin{equation*}
        \mathcal{L}(\hat{y}_i, y_i) = - \left[ y_i \log(\hat{y}_i) + (1 - y_i) \log(1 - \hat{y}_i) \right]
    \end{equation*}
\end{definition}

To find the optimal values of $\theta$, we cannot simply use formula 
like in linear regression since there is no closed-form solution 
for this problem. To circumvent this, we use the gradient decent 
algorithm discussed in \cref{sec:optimization:sub:gradientdecent}.


\section{Deep Learning Basic}

% Explain tasks in Deep Learning

\subsection{Layer}

\subsection{Activation Function}

\subsection{Loss Function}

\section{Optimization}

\subsection{Backpropagation}

\subsection{Gradient Decent}
\label{sec:optimization:sub:gradientdecent}

\subsubsection{Stochastic Gradient Decent}

\subsection{Momentum}

\subsection{Root Mean Square Propagation (RMSProp)}

\subsection{Adaptive Moment Estimation (ADAM)}

\subsection{Newton Method}

\section{Training Neural Network}

\subsection{Data Processing}
% Splitting etc.
% Cross validation

\subsection{Hyperparameter}

\subsection{Common Challenge}

% Underfitting etc.

\subsection{Weight Initialization}

\subsection{Regularization}

\subsubsection{Dropout}

\subsubsection{Batch Normalization}

\subsubsection{Data Augmentation}

\section{Convolutional Neural Network}

\subsection{Filter}

\subsection{Pooling}

\section{Popular CNN Architectures}

\subsection{LeNet}

\subsection{AlexNet}

\subsection{VGGNet}

\subsection{ResNet}
% Introduce skip connection here

\subsection{GoogleNet}
% Introduce Inception Layer here

\subsection{XceptionNet}

\subsection{U-Net}

\section{Sequential Data Deep Learning Architectures}

% Explain what does these model for
% Explain example of examples of sequential data 

\subsection{Recurrent Neural Network (RNN)}

\subsection{Long Short Term Memory (LSTM)}

\subsection{Attention}

\subsection{Transformer}

\section{Generative Deep Learning Architectures}

\subsection{Variational Auto-Encoder}

\subsection{Generative Adversarial Network (GAN)}

\section{Reinforcement Learning}

\section{Closing Remarks}

\end{document}